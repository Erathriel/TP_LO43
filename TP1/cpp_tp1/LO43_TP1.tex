\documentclass[a4paper,11pt]{article}
\usepackage{latexsym}
\usepackage[french]{babel}

%\usepackage[utf8x]{inputenc}
%\usepackage{listings}

\frenchspacing



\begin{document}


%\renewcommand{\labelitemi}{$\bullet$}
%\renewcommand{\labelitemii}{$\bullet$}

\begin{center}
\bfseries\Huge{LO43 - TP1}
\end{center}


\section{Introduction}

\subsection{Quelques informations sur le langage C++}
	
%%%%%%%%%%%%%%%%%%%%%%%%%%%%%%%%%%%%%%%%%%%%%%%%%%%%%%%%%%%%%%%%%%%%%%%%%%%%%%%%%%%%%%%%%%%%%%%%%%%%%%%%%%%%%%%%%%%%%	
%%%%% Header files
%%%%%%%%%%%%%%%%%%%%%%%%%%%%%%%%%%%%%%%%%%%%%%%%%%%%%%%%%%%%%%%%%%%%%%%%%%%%%%%%%%%%%%%%%%%%%%%%%%%%%%%%%%%%%%%%%%%%%	
	
	~\\
		\bfseries
	Fichier d'ent\^ete (header files)
		\normalsize
		\mdseries
	\begin{itemize}
		\item Extension : *.h, *.hpp %(le dernier est pr\'ef\'erable)
		\item D\'eclaration: des class (attributes, m\'ethodes), des noms de fonctions, noms de variables, noms de constantes
		\item Exemple de fichier d'ent\^ete (calculatrice.h): \\
			%		#define PI 3.14
			\begin{verbatim}
			class calculatrice {
			  public:
			    calculatrice() {};
			    ~calculatrice() {};
			    int addition(int a, int b);
			}
			\end{verbatim}
		\item Le fichier peut \'egalement contenir des d\'efinitions de m\'ethodes et de fonctions (recommand\'e seulement pour petite fonction/m\'ethodes)
		\item Les m\'ethodes de la classe d\'efinies dans le corps de la classe sont les m\'ethodes \itshape{inline}. 
	\end{itemize}
	
	
%%%%%%%%%%%%%%%%%%%%%%%%%%%%%%%%%%%%%%%%%%%%%%%%%%%%%%%%%%%%%%%%%%%%%%%%%%%%%%%%%%%%%%%%%%%%%%%%%%%%%%%%%%%%%%%%%%%%%	
%%%%% Source files
%%%%%%%%%%%%%%%%%%%%%%%%%%%%%%%%%%%%%%%%%%%%%%%%%%%%%%%%%%%%%%%%%%%%%%%%%%%%%%%%%%%%%%%%%%%%%%%%%%%%%%%%%%%%%%%%%%%%%		
	
	~\\
		\bfseries
	Fichier source (source files)
		\normalsize
		\mdseries
	\begin{itemize} 
		\item Extension : *.c *.cc, *.cpp %(le dernier est pr\'ef\'erable)
		\item D\'efinir les methodes, fonctions dont le nom est d\'efini dans le fichier d'ent\^ete
		\item Contiennent des r\'ef\'erences aux fichiers d'ent\^ete : \begin{verbatim}#include "..." \end{verbatim}
		\item Exemple de fichier source (calculator.cc):
			\begin{verbatim}
			#include "calculatrice.h"
			
			int calculatrice::addition(int a, int b) {
			  return a+b;
			}			
			\end{verbatim}	
		\item Contiennent une fonction main(...), dont le programme d\'ebute son exploitation (un fichier pour le programme). 
	\end{itemize}
	
	
%%%%%%%%%%%%%%%%%%%%%%%%%%%%%%%%%%%%%%%%%%%%%%%%%%%%%%%%%%%%%%%%%%%%%%%%%%%%%%%%%%%%%%%%%%%%%%%%%%%%%%%%%%%%%%%%%%%%%	
%%%%% Compilation and linking 
%%%%%%%%%%%%%%%%%%%%%%%%%%%%%%%%%%%%%%%%%%%%%%%%%%%%%%%%%%%%%%%%%%%%%%%%%%%%%%%%%%%%%%%%%%%%%%%%%%%%%%%%%%%%%%%%%%%%%		

	~\\	
			\bfseries
	La compilation et le linkage
			\normalsize
			\mdseries
		\begin{itemize} 
			\item La compilation: le compilateur compile chaque fichier source (.c, .cc .cpp), c'est-\`a-dire qu'il cr\'ee un fichier binaire (.o) par fichier source. Exemple: % (except\'e pour le fichier contenant la fonction main.) 
				\begin{verbatim}
				g++ -c fichier.cpp -I/chemin
				\end{verbatim}
				%gcc -c fichier.cpp -I/chemin
				\begin{itemize}
					\item fichier.cpp - fichier source fichier.cc, qui sera compil\'e (cr\'ee fichier fichier.o)
					\item -L/chemin - ajouter le r�pertoire /chemin � la liste des r�pertoires o� aller chercher les fishier d'ent\^ete
					%\item -I/chemin - chemin ou sont les fichiers d'ent\^ete inclus dans fichier.cc ( si les fichiers d'ent\^ete ce trouve dans le m\^eme dossier que fichier.cc, on ne met pas le chemin )
			\end{itemize}
			~\\
			\item Le linkage: Cette phase, constitue la phase finale pour obtenir un ex\'ecutable. Le compilateur agr�ge chaque fichier .o avec les \'eventuels fichiers binaires des librairies qui sont utilis\'ees (fichiers .a et .so sous linux, fichiers .dll et .lib sous windows). Exemple:
			\begin{verbatim}
				g++ -o programme programme.o binaires1.o . . . binairesN.o -L/chemin -lsunmath
			\end{verbatim}
				% gcc -o programme programme.o binaires1.o . . . binairesN.o -L/chemin -lsunmath
			
			\begin{itemize}
				\item programme.o - fichier binaire de fichier source programme.cc contiennent fonction main(...)
				\item binaires1.o - fichier binaire de fichier source binaires1.cc
				%\item -L. - chemin ou sont les librairies utilis\'e pour le linkage ( si les librairies ce trouve dans standard 	 dossier que fichier.cc, on ne met pas le chemin )
				\item -L/chemin - ajouter le r�pertoire /chemin � la liste des r�pertoires o� aller chercher les librairies
				\item -lsunmath - librairie sunmath
			\end{itemize}
		\end{itemize}
		

%%%%%%%%%%%%%%%%%%%%%%%%%%%%%%%%%%%%%%%%%%%%%%%%%%%%%%%%%%%%%%%%%%%%%%%%%%%%%%%%%%%%%%%%%%%%%%%%%%%%%%%%%%%%%%%%%%%%%	
%%%%% Makefile
%%%%%%%%%%%%%%%%%%%%%%%%%%%%%%%%%%%%%%%%%%%%%%%%%%%%%%%%%%%%%%%%%%%%%%%%%%%%%%%%%%%%%%%%%%%%%%%%%%%%%%%%%%%%%%%%%%%%%		
		
			~\\	
			\bfseries
	Makefile
			\normalsize
			\mdseries

	Makefile est un fichier pour le programme make. Le programme make analyse un fichier Makefile et des r\`egles sur cette base que les fichiers source doivent compiler. Il permettre de gagner de temps dans la cr\'eation du programme, car \`a la suite de modifications dans le fichier source uniquement les fichiers qui sont d\'ependant de ce fichier sont compil\'e.
	
	Makefile se compose d'un ensemble de r\`egles. R\`egles consistent g\'en\'eralement de cible, des r\`egles de d\'ependance et des commandes. On peux aussi d\'efinir de variable utilises dans le r\`egles. Pour pour ex\'ecuter Makefile tape "make".
	
%	sk?ada sie; z zestawu regu?. Regu?y sk?adaja; sie; zazwyczaj z celu (nazwy celu i zazwyczaj zarazem nazwy pliku jaki zostanie wygenerowany za pomoca; danej regu?y), zalez.nos'ci (moga; to byc' nazwy istnieja;cych plik�w jak i nazwy innych regu?) oraz zestawu komend kt�re maja; byc' wykonane przy danej regule (zazwyczaj tworza; one plik celu). Ponadto w Makefile moz.na deklarowac' zmienne i sie; nimi pos?ugiwac'. Do wartos'ci zmiennych odwo?ujemy sie; poprzedzaja;c je znakiem $ i biora;c nazwe; w nawiasy okra;g?e (nawiasy pomija sie; przy zmiennych jednoznakowych). 
	
	%http://www.fuw.edu.pl/~jnareb/zajecia/make.html
	
	
	
	%# Le variable qui contient chemins ou se trouvent les fichiers a inclure
	%INCLUDES= -I/usr/include
	
	Exemple :
	\begin{verbatim}
	# C'est un commentaire
	
	# Le variable qui indique le compilateur
	CXX=g++

	# Le variable avec les r�pertoires ou ce trouve fichiers d'entetes
	INCLUDE_DIRS=-I/usr/include

	# Le variable avec les r�pertoires ou ce trouve librairies
	LIB_DIRS=-L/usr/lib

	# Le variable avec les options de compilation	CXXFLAGS= $(INCLUDE_DIRS) -c 

	# Le variable avec les options de linkage	LDFLAGS= $(LIB_DIRS)
		
	# Le variable qui contient les librairies avec lesquelles on va 
	# effectue l'edition de liens
	LIBS=-lm

	# Le cible:
	# <nome_de_cible>: <dependance> 
	#   <commandes>
	
	# Le cible qui va cree le executable
	tp: programme.o func1.o func2.o
		$(CXX) $(LIB_DIRS) -o main programme.o func1.o func2.o $(LIBS)
	
	# Le cible qui supprime tout les fichiers interm\'ediaires.
	clean:
  rm -f func1.o func2.o programme
	
	
	# Les regles de dependance 
	# <nome_de_regle>: <dependance>    
	#   <commandes>
	programme.o: programme.c func3.h
    $(CXX) $(CXXFLAGS) programme.c
    
  func1.o: func1.c
    $(CXX) $(CXXFLAGS) func1.c
  
  func2.o: func2.c
    $(CXX) $(CXXFLAGS) func2.c
	\end{verbatim}
		
	
	
	\subsection{Les Resources}
		\begin{itemize}
			\item C++ FAQ LITE : \underline{http://jlecomte.ifrance.com/c++/c++-faq-lite/index-fr.html}
			\item Introduction \`a Makefile : \underline{http://gl.developpez.com/tutoriel/outil/makefile/}
			\item gnu 'make' : \underline{http://www.gnu.org/software/autoconf/manual/make/index.html}
			\item Manuel de gcc : \itshape{man gcc}
		\end{itemize}
	

	
%%%%%%%%%%%%%%%%%%%%%%%%%%%%%%%%%%%%%%%%%%%%%%%%%%%%%%%%%%%%%%%%%%%%%%%%%%%%%%%%%%%%%%%%%%%%%%%%%%%%%%%%%%%%%%%%%%%%%	
%%%%% 
%%%%% Excercises
%%%%%
%%%%%%%%%%%%%%%%%%%%%%%%%%%%%%%%%%%%%%%%%%%%%%%%%%%%%%%%%%%%%%%%%%%%%%%%%%%%%%%%%%%%%%%%%%%%%%%%%%%%%%%%%%%%%%%%%%%%%		
	

\section{Excercises}

\subsection{Helloworld}
~\\
				1) \'Ecrire un programme qui lance une m\'ethode \itshape{affiche} de classe \itshape{HelloW} affichant sur un \'ecran "HelloW - hello world !".\\
					~\\
					HelloW.hpp : d\'eclaration de la classe \itshape{HelloW}\\
					HelloW.cpp : impl\'ementation de la classe \itshape{HelloW}\\
					ClientHW.cpp : programme de test (contient la m\'ethode main)\\
					~\\
					Compilation et linkage avec gcc\\
					~\\
%		\begin{enumerate}
%			\item \'Ecrire un programme qui lance une m\'ethode de classe HelloW affichant sur un \'ecran <<HelloW - hello world !>>.
%				\begin{itemize} %{labelitemi}{$\bullet$}
%					\item Fichier d'ent\^ete avec d\'eclaration de la classe %plik naglowkowy .h ( deklaracja klasy )
%					\item Fichier source avec impl\'ementation de la m\'ethodes de la classe %plik zrodlowy .cc ( definicja klasy } 
%					\item Fichier source avec m\'ethode main %plik z main  .cc 
%					\item Compilation et linkage avec gcc  % kompilacja przy uzyciu gcc 
%				\end{itemize}
%			\item 
				2) Ajoute les classes \itshape{HelloA}, \itshape{HelloB} avec les methodes \itshape{affiche} qui affiche sur un ecran "HelloA - hello world!" et "HelloB - hello world!".\\
				~\\
				Pour chaque classe une fichier d'ent\^ete et fichier source.\\
				Compilation avec un fichier makefile.\\
			
%			\item Ajoute les classes HelloA, HelloB avec les methodes qui affiche sur un ecran <<HelloA - hello world!>> et <<HelloB - hello world!>>.
%				\begin{itemize}
%					\item Pour chaque classe une fichier d'ent\^ete et fichier source.
%					\item Compilation avec un fichier makefile
%				\end{itemize}
%			\item 
%		\end{enumerate}
		
\subsection{Type Abstrait Vecteur}
~\\
			L'objet du TP est de construire le type vecteur et de le tester avec un programme client.\\
~\\			
			1) Compilation: rapatrier dans votre r\'epertoire de travail une copie du dossier cpp\_tp1 qui contient
un squelette de programme et un makefile:\\
~\\
	Vecteur.h : interface du TAD\\ % TAD = Type Abstraite de Donne
	Vecteur.cc : impl\'ementation\\
	Client.cc : programme de test du TAD (contient le main)\\ 
	Makefile : permet la compilation et l'\'edition de liens du programme\\
	Apr\`es avoir copier les fichiers, tester le compilation en utilisant Makefile.\\
~\\	
	2) Construction du TAD\\ 
~\\	
Compl\'eter les fichiers Vecteur.h et Vecteur.cc avec les op\'erateurs demand\'es.\\
Proc\'eder par \'etapes en v\'erifiant la syntaxe (compilation) au fur et a mesure.\\
Pour les diff\'erents constructeurs/destructeur on d\'efinira des compteurs statiques
pour calculer le nombre de passages dans chacun d'eux.\\
~\\
3) Programme client\\
Compl\'eter le programme client avec des exemples d'utilisation du TAD
			
			

\end{document}


